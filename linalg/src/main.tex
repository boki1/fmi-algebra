\documentclass[a4paper]{article}

\usepackage[bulgarian]{babel}
\usepackage[utf8]{inputenc}
\usepackage{csquotes}
\usepackage{amsmath}
\usepackage{amsthm}
\usepackage{amsfonts}
\usepackage[framemethod=TikZ]{mdframed}
\usepackage[colorlinks=true, allcolors=blue]{hyperref}
\usepackage[nottoc]{tocbibind}
\usepackage{titlesec}
\usepackage[letterpaper,top=2cm,bottom=2cm,outer=4cm,inner=2cm,heightrounded,marginparwidth=3cm,
marginparsep=.7cm]{geometry}
\usepackage{ragged2e}
\usepackage[acronym,toc]{glossaries}
\usepackage{wasysym}

\usepackage{xcolor}
\definecolor{superlightorange}{HTML}{fff0e6}
\definecolor{superlightyellow}{HTML}{ffffcc}
\definecolor{superlightpurple}{HTML}{ffe6ff}
\definecolor{darkgrey}{HTML}{666666}

\usepackage{graphicx}
\graphicspath{{images/}}

\usepackage[style=alphabetic,backend=biber]{biblatex}
\addbibresource{literature.bib}

\usepackage{marginnote}
\renewcommand*{\marginfont}{\normalfont\scriptsize\rmfamily}
\newcommand{\sidenote}[2][0cm] {
	\marginnote{
		\begin{mdframed}[rightline=false,topline=false,bottomline=false,leftmargin=-.5cm,outerlinewidth=.7,
			innertopmargin=0,innerbottommargin=0,innerleftmargin=4,backgroundcolor=white]
			\begin{flushleft}
				#2
			\end{flushleft}
		\end{mdframed}
	}[#1]
}

\newtheorem*{fact}{Факт}
\newmdtheoremenv[backgroundcolor=superlightorange,innertopmargin=0pt]{theorem}{Теорема}[section]
\newmdtheoremenv[backgroundcolor=superlightpurple,innertopmargin=0pt]{crly}{Следствие}[theorem]
\newmdtheoremenv[backgroundcolor=superlightyellow,innertopmargin=0pt]{lemma}{Лема}[section]
\newtheorem{definition}{Определение}[section]

\titleformat{\section}{\normalfont\bfseries\large}{\thesection.}{1em}{}

\newcommand{\nullvec}[0]{\mathbf{0}}

\newcommand{\dfn}[1]{\underline{#1}}
\renewcommand{\subsection}[1]{\noindent\rule{\textwidth}{.7pt} \hfill \begin{center}\wasyparagraph \hspace{1em}
\textbf{#1} \end{center} }
\titlespacing*{\subsection} {0pt}{2.25ex plus 1ex minus .2ex}{0.5ex plus .2ex}
\newcommand{\prgraph}[1]{}

\setglossarystyle{list}
\makeglossaries
\newacronym{vecsp}{ЛП}{Линейно пространство}
\newacronym{lin-depen}{ЛЗ}{Линейно зависими}
\newacronym{lin-indepen}{ЛНЗ}{Линейно независими}
\newacronym{fund-sys-sol}{ФСР}{Фундаментална система решения}

\title{\textbf{Резултати по линейна алгебра}}
\author{Кристиян Стоименов}
\date{2024}

\begin{document}

\maketitle

\renewcommand{\abstractname}{}
\begin{abstract}
\par \textsc{Настоящите} записки ``Резултати по линейна алгебра'' не предяват оригиналност, а изцяло целят подпомагане на
съставителя в усвояването на материала. Съставени са на база курса, четен във
ФМИ, и учебника, използван там \autocite{fmi-linalg-notes}.
\end{abstract}

\tableofcontents
\bigskip
\printglossary[type=\acronymtype, toctitle=Използвани съкращения,title=Използвани съкращения]
\newpage

\section{Детерминанти}

\section{Линейни пространства}

\subsection{Основни понятия}

\begin{definition} \label{dfn:vectorspace}
Нека $F$ е поле и $V$ е непразно множество, чиито елементи ще наричаме
вектори. Нека във $V$ са въведени следните операции:
\begin{itemize}
\item Събиране на вектори: $\forall a, b \in V \implies a + b \in V$;
\item Умножение на вектор със скалар от полето: $\forall a \in V, \forall \lambda
	\in F \implies \lambda a \in V$
\end{itemize}
Ще казваме, че $V$ е \dfn{\acrfull{vecsp} над полето $F$}, ако тези операции
удовлетворяват следните свойства (аксиоми):
\begin{itemize}
\item Комутативност на събирането: $\forall a, b \in V \implies a + b = b + a$
\item Асоциативност при събирането: $\forall a,b,c \in V \implies a + (b + c) =
	(a + b) + c$
\item Неутрален елемент (спрямо събирането): $\exists \nullvec \in V, \forall a
	\in V \implies a + \nullvec = \nullvec + a = a$
\item Противоположен елемент: $\forall a \in V \exists a' \in V$, т-че $a + a' =
	a' + a = \nullvec$
	\sidenote{Противоположният елемент на $a$ обичайно бележим $-a$.}
\item Умножение с $1 \in F$: $\forall a \in V \implies 1.a = a$
\item Дистрибутивност при събиране на скалари: $\forall \lambda, \mu \in F,
	\forall a \in V \implies (\lambda + \mu)a = \lambda a + \mu a$
\item Дистрибутивност при събиране на вектори: $\forall \lambda \in F,
	\forall a, b \in V \implies \lambda(a + b) = \lambda a + \lambda b$
\item Асоциативност на умножението със скалар: $\forall \lambda, \mu \in F,
	\forall a \in V \implies \lambda(\mu a) = (\lambda \mu)a$
\end{itemize}
\end{definition}

\begin{theorem}[Основни свойства на \acrshort{vecsp}]

Нека $V$ е линейно пространство над полето $F$. Тогава са изпълнени следните
свойства:

\begin{itemize}
\item Единственост на нулевия вектор.
\item Единственост на противоположния елемент.
\item $\forall a \in V \implies 0.a = \nullvec$
\item $\forall \lambda \in F \implies \lambda\nullvec = \nullvec$
\item $\forall a \in V \implies (-1)a = -a$
\item Ако $\lambda \in F, a \in V$ и $\lambda a = \nullvec$, то или $\lambda = 0$, или $a = \nullvec$.
\item Ако $a, b \in V$, то уравнението $a + x = b$ има единствено решение във $V$.
	\sidenote[-1cm]{Това решение оттук нататък вместо $b + (-a)$ ще пишем $b - a$.}
\end{itemize}
\end{theorem}

\begin{definition} \label{dfn:lin-combination}
Нека $V$ е \acrshort{vecsp} над полето $F$. Ако $a_1, a_2, \dots, a_n \in V$ и
$\lambda_1, \lambda_2, \dots, \lambda_n \in F$, то вектора $\lambda_1 a_1 +
\lambda_2 a_2 + \dots + \lambda_n a_n \in V$ наричаме \dfn{линейна комбинация} на
векторите $a_i$ с коефициенти $\lambda_i$. Множеството от всички линейни
комбинации на векторите $a_1, a_2, \dots, a_n$ с коефициенти от полето $F$
наричаме \dfn{линейна обвивка} на $a_1, a_2, \dots, a_n$.
	\sidenote[-.8cm]{Бележим $l(a_1, a_2, \dots, a_n)$.}
\end{definition}

\begin{definition} \label{dfn:lin-dependence}
Нека $V$ е \acrshort{vecsp} над полето $F$. Ако $a_1, a_2, \dots, a_n \in V$, то
тях ще наричаме \dfn{\acrfull{lin-depen}}, ако съществуват такива коефициенти
$\lambda_1, \lambda_2, \dots, \lambda_n \in F$, че \[
	\lambda_1 a_1 + \lambda_2 a_2 + \dots + \lambda_n a_n = \nullvec
\]
В противен случай - ако такива коефициенти не съществуват, то за векторите ще
казваме, че са \dfn{\acrfull{lin-indepen}}.
\end{definition}

\begin{theorem}[Основни свойства на \acrshort{lin-depen} и \acrshort{lin-indepen}
	вектори]\label{thm:lin-dependence}
\begin{itemize}
\item Един вектор е \acrshort{lin-depen} точно когато е нулевият вектор.
\item Два вектора са \acrshort{lin-depen} точно когато са пропорционални.
\item Ако една система от вектори съдържа нулевия вектори или два пропорционални, то тя е \acrshort{lin-depen}.
\item Всяка подсистема на \acrshort{lin-indepen} система от вектори също е \acrshort{lin-indepen}.
\item Една система от поне два вектора е \acrshort{lin-depen} точно когато поне един от вектори
	може да се представи като линейна комбинация на останалите вектори от системата.
\end{itemize}
\end{theorem}

\begin{lemma}[Основна лема на линейната алгебра] \label{lma:fundamental-lemma}
Нека $V$ е \acrshort{vecsp} над полето $F$. Ако имаме два системи от вектори \[
	A = { a_1, a_2, \dots, a_n }, \\
	B = { b_1, b_2, \dots, b_k}
\] за които знаем, че всеки вектор от $B$ може да се представи като линейна
комбинация на векторите от $A$ и освен това $k > n$, то векторите от системата
$B$ са \acrshort{lin-depen}.
	\sidenote[-1.6cm]{\underline{т.е} ако повече на брой вектори се изразяват линейно чрез
		по-малко на брой вектори, то повечето на брой вектори са \acrshort{lin-depen}.}
\end{lemma}

\begin{definition} \label{dfn:vector-subspace}
Нека $V$ е \acrshort{vecsp} и $W$ е непразно подмножество, $\emptyset \neq W
\subseteq V$. Ще казваме, че $W$ е линейно \dfn{подпространство} на $V$,
ако всяка линейна комбинация на вектори от $W$ също принадлеци на $W$.
\par Еквивалентно, $W$ е подпространство на $V$, ако е затворено относно
операциите събиране и умножение със скалар.
\sidenote[-.5cm]{Бележим $W < V$.}
\end{definition}

\begin{theorem}[Основни свойства на подпространствата] \label{thm:vector-subspace}
Следните свойства са изпълнени:
\begin{itemize}
\item Всяко подпространство съдържа нулевия вектор.
\item Сечение на подпространства е подпространство.
\item Обединение на подпространства е подпространство само когато едното е
	помножество на другото.
\item Ако $A$ е система от вектори, то $l(A)$ е подпространство (и то
	минималното по включване такова).
\end{itemize}
\end{theorem}

\begin{lemma}
Нека $V$ е \acrshort{vecsp} и $a_1, a_2, \dots, a_n \in V$ са
\acrshort{lin-indepen}. Ако $b \in V$ и $b \notin l(a_1, a_2, \dots, a_n)$, то
$a_1, a_2, \dots, a_n, b$ също са \acrshort{lin-indepen}.
\end{lemma}

\subsection{Базис и размерност}

\begin{definition} \label{dfn:basis}
Нека $V \neq \emptyset$ е \acrshort{vecsp} над полето $F$ и $\emptyset \neq B \subseteq V$.
Ще казваме, че $B$ е \dfn{базис} на $V$, ако е изпълнено, че:
\begin{itemize}
\item $B$ е \acrshort{lin-indepen} система от вектори и
\item $l(B) = V$
\end{itemize}
\par Ако $B$ е крайно множество, то пространството $V$ ще наричаме \dfn{крайномерно}, а
иначе - \dfn{безкрайномерно}.
\end{definition}

\begin{theorem}
Ако $A = a_1, a_2, \dots, a_n$ и $V = l(A)$, то съществува подсистема на $A$,
която е базис на $V$ над $F$.
\end{theorem}

\begin{theorem}
Всеки два базиса на крайномерно пространство над поле $F$ съдържат равен брой
вектори.
\end{theorem}

\begin{definition}
Нека $V$ е \acrshort{vecsp} над полето $F$. Тогава броят $k$ на векторите в кой
да е базис на $V$ наричаме \dfn{размерност} на $V$ над $F$. В случая, когато $V$ е
безкрайномерно, то размерността е $\infty$.
	\sidenote{Размерност бележим $dim_FV$.}
\end{definition}

\begin{theorem}
Нека $V$ е \acrshort{vecsp} над полето $F$. Тогава:
\begin{itemize}
\item $V$ е крайномерно ($dimV=n$) точно когато във $V$ съществуват $n$ на брой
	\acrshort{lin-indepen} вектори и всеки $n+1$ от са \acrshort{lin-depen}.
	\sidenote[-1cm]{В такъв случай всеки $n$ \acrshort{lin-indepen} вектора са базис на $V$.}
\item $V$ е безкрайномерно ($dimV = \infty$) точно когато за всяко естествено
	число $n \in \mathbb{N}$ можем да намерим вектори $a_1, a_2, \dots, a_n \in
	V$, които да са \acrshort{lin-indepen}.
\end{itemize}
\end{theorem}

\begin{theorem}
	Ако $V$ е крайномерно \acrshort{vecsp}, то всяка \acrshort{lin-indepen} система
	от вектори от $V$ може да се допълни до базис на $V$.
\end{theorem}

\begin{theorem}
Нека $V$ е крайномерно пространство и $W < V$. Тогава:
\begin{itemize}
\item $W$ също е крайномерно и $dimW \leq dimV$ и
\item $dimW = dimV \iff W = V$
\end{itemize}
\end{theorem}

\begin{theorem}
Нека $V$ е крайномерно \acrshort{vecsp}. Една система $B \subseteq V$ е базис
точно когато всеки вектор от $V$ се изразява линейно чрез $B$ по единствен
начин.
\end{theorem}

\begin{definition}
Нека $V$ е крайномерно \acrshort{vecsp} над полето $F$ и ${b_1, b_2, \dots,
b_n}$ е фиксиран базис на $V$. Тогава, ако $u \in V$ и \[
	u = \lambda_1 b_1 + \lambda_2 b_2 + \dots + \lambda_n b_n,
\]
то числата $\lambda_1, \lambda_2, \dots, \lambda_n$ наричаме \dfn{координати} на $u$
в базиса $b_1, b_2, \dots, b_n$.
\end{definition}

\subsection{Сума на подпространства}

\begin{definition}
Нека $V_1, V_2, \dots, V_n$ са подпространства. \dfn{Сума} на подпространства
ще наричаме \[
	V := V_1 + V_2 + \dots + V_n = \{ v = v_1 + v_2 + \dots + v_n \mid v_i \in V_i,
	i=1,\dots,n \}
\]
Ако всеки вектор $v \in V$ се преставя по единствен начин като сума на вектори
от $V_i$, то казваме, че $V$ е \dfn{директна сума} на $V_1, V_2, \dots, V_n$.
	\sidenote[-1cm]{Директна сума бележим $V = V_1 \oplus V_2 \oplus \dots V_n$.}
\end{definition}

\begin{theorem}
	Нека $V$ е крайномерно \acrshort{vecsp} и $V_1 < V$, $V_2 < V$. Тогава е
	изпълнено следното равенство: \[
		dim(V_1 + V_2) = dimV_1 + dimV_2 - (dimV_1 \cap dimV_2)
	\]
\end{theorem}

\begin{theorem}
	Нека $V$ е крайномерно \acrshort{vecsp} и $V_1 < V$, $V_2 < V$. Тогава \[
		V = V_1 \oplus V_2 \iff
		\begin{cases}
			V = V_1 + V_2 \\
			V_1 \cap V_2 = \nullvec
		\end{cases}
	\]
\end{theorem}
	\sidenote[-2.3cm]{Твърдението ествествно се обобщава и сума от произволено
	брой подпространства.}

\begin{theorem}
Нека $V, dimV = n$ е \acrshort{vecsp} с базис $e_1, e_2, \dots, e_n$. Ако $k
\leq n$, то $V$ се преставя като $V = V_1 \oplus V_2$, където $V_1 = l(e_1, \dots, e_k)$
и $V_2 = l(e_{k+1}, \dots, e_n)$.
\end{theorem}
	\sidenote[-2.2cm]{Ясно е, че $e_1, \dots, e_k$ и $e_{k+1}, \dots, e_n$ се явяват базиси
	съотвено на $V_1$ и $V_2$, което пък влече $dimV_1 = k, dimV_2 = n - k$.}

\begin{theorem}
Нека $V, dimV = n$ е \acrshort{vecsp} и $W < V$. Тогава съществува $U < V$,
такова че $V = W \oplus U$.
\end{theorem}

\begin{fact}
Нека $V, dimV = n$ е \acrshort{vecsp}. Тогава, ако $e_1, \dots, e_n$ е базис на
$V$, то $V$ е директна сума на едномерни подпространства - $V = l(e_1) \oplus l(e_2) \oplus \dots \oplus l(e_n)$
\end{fact}

\subsection{Ранг на система вектори. Ранг на матрица.}

\begin{definition}
\par Нека имаме матрица $A$. Ако фиксираме произволни $k$ реда и $k$ стълба, то те се
пресичат, така че да образуват квадратна матрица от ред $k$. Детерминантата на
на всяка такава матрица ще наричаме \dfn{минор} на $A$ от ред $k$.
\\ \smallskip
Ще казваме, че матрицата $A$ има \dfn{ранг} $r$, ако притежава различен от нула
митор от ред $r$, а всички минори от по-голям ред са равни на нула.
\sidenote[-1cm]{Ранг на матрица бележим $r(A)$.}
\end{definition}

\par Понятието \textit{ранг} може да се изрази и чрез система вектори.

\begin{definition}
Нека $V$ е \acrshort{vecsp} и $c_1, c_2, \dots, c_n \in V$ е система от вектори.
Казваме, че системата има \dfn{ранг} $r$, ако $r$ от тези $n$ вектора са
\acrshort{lin-indepen}, а всеки $n+1$ са задължително \acrshort{lin-depen}.
\end{definition}

\sidenote[-.3cm]{Съществува връзка между понятията \textit{размерност} на линейно
	подпространство и \textit{ранг} на система от вектори, която се изразява по
	следния начин:
		\begin{equation*}
		\begin{split}
			r&(c_1, \dots, c_n) = \\=  dim l&(c_1, \dots, c_n)
		\end{split}
		\end{equation*}
	}

\begin{theorem}
	Нека $A_{m\times n}$ е матрица с елементи от полето $F$. Тогава, ако отбележим
вектор-редовете на $A$ с $a_1, \dots, a_m$ и вектор-стълбовете с $b_1, \dots,
b_n$, то е изпълнено и равенството \[
	r(a_1, \dots, a_m) = \\ r(b_1, \dots, b_n) = \\ r(A)
\]
\end{theorem}

\begin{crly}
Ако $А$ квадратна матрица, то $detA = 0$ точко когато редовете (или стълбовете)
на $A$ са \acrshort{lin-depen}.
\end{crly}

\subsection{Хомогенни системи}

\begin{theorem}[Теорема на Руше-Кронекер-Капели]
Ако $A$ и $\bar{A}$ са съответно матрицата и разширената матрица на линейна
система, то тя е съвместима точко когато $r(A) = r(\bar{A})$.
\end{theorem}

\begin{theorem}
Нека $A$ e матрицата на линейна система. Ако тя e квадратна, то тогава системата
има ненулево решение точно когато $detA = 0$.
\end{theorem}

\begin{fact}
Множеството от решенията на хомогенна система е подпространство на $F^n$.
\end{fact}

\begin{definition}
Всеки базис на пространството от решения, ако то е ненулево, на хомогенната
система ще наричаме \acrfull{fund-sys-sol} на тази система.
	\sidenote[-2cm]{Така всяка \acrshort{fund-sys-sol} се състои от
		\acrshort{lin-indepen} вектори от $F^n$, които са решения на
		хомогненната система и всяко друго решение е тяхна линейна комбинация.}
\end{definition}

\begin{theorem}
Нека $U$ е подпространството от решенията на хомогенна ситемата. Тогава, ако
$r(A) = r$, то $dimU = n - r$.
	\sidenote[-1cm]{\underline{т.е} всяка \acrshort{fund-sys-sol} се състои от точно
		$n - r$ на брой решения.}
\end{theorem}

\begin{theorem}
Всяко подпространство $W$ на $F^n$ е пространството от решения на подходяща
хомогенна система линейни уравнения с $n$ неизвестни.
\end{theorem}

\section{Линейни изображения}

\newpage
\section{Литература}
\nocite{*}
\printbibliography[heading=none]

\end{document}
