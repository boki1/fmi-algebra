\documentclass[a4paper]{article}

\usepackage[bulgarian]{babel}
\usepackage[utf8]{inputenc}
\usepackage{csquotes}
\usepackage{amsmath}
\usepackage{amsthm}
\usepackage{amsfonts}
\usepackage[framemethod=TikZ]{mdframed}
\usepackage[colorlinks=true, allcolors=blue]{hyperref}
\usepackage[nottoc]{tocbibind}
\usepackage{titlesec}
\usepackage[letterpaper,top=2cm,bottom=2cm,outer=4cm,inner=2cm,heightrounded,marginparwidth=3cm,
marginparsep=.7cm]{geometry}
\usepackage{ragged2e}
\usepackage[acronym,toc]{glossaries}

\usepackage{xcolor}
\definecolor{superlightorange}{HTML}{fff0e6}
\definecolor{superlightyellow}{HTML}{ffffcc}
\definecolor{darkgrey}{HTML}{666666}

\usepackage{graphicx}
\graphicspath{{images/}}

\usepackage[style=alphabetic,backend=biber]{biblatex}
\addbibresource{literature.bib}

\usepackage{marginnote}
\renewcommand*{\marginfont}{\normalfont\scriptsize\rmfamily}
\newcommand{\sidenote}[2][0cm] {
	\marginnote{
		\begin{mdframed}[rightline=false,topline=false,bottomline=false,leftmargin=-.5cm,outerlinewidth=.7,
			innertopmargin=0,innerbottommargin=0,innerleftmargin=4,backgroundcolor=white]
			\begin{flushleft}
				#2
			\end{flushleft}
		\end{mdframed}
	}[#1]
}

\newtheorem*{fact}{Факт}
\newmdtheoremenv[backgroundcolor=superlightorange,innertopmargin=0pt]{theorem}{Теорема}[section]
\newmdtheoremenv[backgroundcolor=superlightyellow,innertopmargin=0pt]{lemma}{Лема}[section]
\newtheorem{definition}{Определение}[section]

\titleformat{\section}{\normalfont\bfseries\large}{\thesection.}{1em}{}

\newcommand{\nullvec}[0]{\mathbf{0}}

\glossarystyle{list}
\makeglossaries
\newacronym{vecsp}{ЛП}{Линейно пространство}
\newacronym{lin-depen}{ЛЗ}{Линейно зависими}
\newacronym{lin-indepen}{ЛНЗ}{Линейно независими}

\title{\textbf{Резултати по линейна алгебра}}
\author{Кристиян Стоименов}
\date{2024}

\begin{document}

\maketitle

\renewcommand{\abstractname}{}
\begin{abstract}
\par \textsc{Настоящите} записки ``Резултати по линейна алгебра'' не предяват оригиналност, а изцяло целят подпомагане на
съставителя в усвояването на материала. Съставени са на база курса, четен във
ФМИ, и учебника, използван там \autocite{fmi-linalg-notes}.
\end{abstract}

\tableofcontents
\bigskip
\printglossary[type=\acronymtype, toctitle=Използвани съкращения,title=Използвани съкращения]
\newpage

\section{Линейни пространства}

\begin{definition} \label{dfn:vectorspace}
Нека $F$ е поле и $V$ е непразно множество, чиито елементи ще наричаме
вектори. Нека във $V$ са въведени следните операции:
\begin{itemize}
\item Събиране на вектори: $\forall a, b \in V \implies a + b \in V$;
\item Умножение на вектор със скалар от полето: $\forall a \in V, \forall \lambda
	\in F \implies \lambda a \in V$
\end{itemize}
Ще казваме, че $V$ е \acrfull{vecsp} над полето $F$, ако тези операции
удовлетворяват следните свойства (аксиоми):
\begin{itemize}
\item Комутативност на събирането: $\forall a, b \in V \implies a + b = b + a$
\item Асоциативност при събирането: $\forall a,b,c \in V \implies a + (b + c) =
	(a + b) + c$
\item Неутрален елемент (спрямо събирането): $\exists \nullvec \in V, \forall a
	\in V \implies a + \nullvec = \nullvec + a = a$
\item Противоположен елемент: $\forall a \in V \exists a' \in V$, т-че $a + a' =
	a' + a = \nullvec$
	\sidenote{Противоположният елемент на $a$ обичайно бележим $-a$.}
\item Умножение с $1 \in F$: $\forall a \in V \implies 1.a = a$
\item Дистрибутивност при събиране на скалари: $\forall \lambda, \mu \in F,
	\forall a \in V \implies (\lambda + \mu)a = \lambda a + \mu a$
\item Дистрибутивност при събиране на вектори: $\forall \lambda \in F,
	\forall a, b \in V \implies \lambda(a + b) = \lambda a + \lambda b$
\item Асоциативност на умножението със скалар: $\forall \lambda, \mu \in F,
	\forall a \in V \implies \lambda(\mu a) = (\lambda \mu)a$
\end{itemize}
\end{definition}

\begin{theorem}[Основни свойства на \acrshort{vecsp}]

Нека $V$ е линейно пространство над полето $F$. Тогава са изпълнени следните
свойства:

\begin{itemize}
\item Единственост на нулевия вектор.
\item Единственост на противоположния елемент.
\item $\forall a \in V \implies 0.a = \nullvec$
\item $\forall \lambda \in F \implies \lambda\nullvec = \nullvec$
\item $\forall a \in V \implies (-1)a = -a$
\item Ако $\lambda \in F, a \in V$ и $\lambda a = \nullvec$, то или $\lambda = 0$, или $a = \nullvec$.
\item Ако $a, b \in V$, то уравнението $a + x = b$ има единствено решение във $V$.
	\sidenote[-1cm]{Това решение оттук нататък вместо $b + (-a)$ ще пишем $b - a$.}
\end{itemize}
\end{theorem}

\begin{definition} \label{dfn:lin-combination}
Нека $V$ е \acrshort{vecsp} над полето $F$. Ако $a_1, a_2, \dots, a_n \in V$ и
$\lambda_1, \lambda_2, \dots, \lambda_n \in F$, то вектора $\lambda_1 a_1 +
\lambda_2 a_2 + \dots + \lambda_n a_n \in V$ наричаме лиейна комбинация на
векторите $a_i$ с коефициенти $\lambda_i$. Множеството от всички линейни
комбинации на векторите $a_1, a_2, \dots, a_n$ с коефициенти от полето $F$
наричаме линейна обвивка на $a_1, a_2, \dots, a_n$.
	\sidenote[-.8cm]{Бележим $l(a_1, a_2, \dots, a_n)$.}
\end{definition}

\begin{definition} \label{dfn:lin-dependence}
Нека $V$ е \acrshort{vecsp} над полето $F$. Ако $a_1, a_2, \dots, a_n \in V$, то
тях ще наричаме \acrfull{lin-depen}, ако съществуват такива коефициенти
$\lambda_1, \lambda_2, \dots, \lambda_n \in F$, че \[
	\lambda_1 a_1 + \lambda_2 a_2 + \dots + \lambda_n a_n = \nullvec
\]
В противен случай - ако такива коефициенти не съществуват, то за векторите ще
казваме, че са \acrfull{lin-indepen}.
\end{definition}

\begin{theorem}[Основни свойства на \acrshort{lin-depen} и \acrshort{lin-indepen}
	вектори]\label{thm:lin-dependence}
\begin{itemize}
\item Един вектор е \acrshort{lin-depen} точно когато е нулевият вектор.
\item Два вектора са \acrshort{lin-depen} точно когато са пропорционални.
\item Ако една система от вектори съдържа нулевия вектори или два пропорционални, то тя е \acrshort{lin-depen}.
\item Всяка подсистема на \acrshort{lin-indepen} система от вектори също е \acrshort{lin-indepen}.
\item Една система от поне два вектора е \acrshort{lin-depen} точно когато поне един от вектори
	може да се представи като линейна комбинация на останалите вектори от системата.
\end{itemize}
\end{theorem}

\begin{lemma}[Основна лема на линейната алгебра] \label{lma:fundamental-lemma}
Нека $V$ е \acrshort{vecsp} над полето $F$. Ако имаме два системи от вектори \[
	A = { a_1, a_2, \dots, a_n }, \\
	B = { b_1, b_2, \dots, b_k}
\] за които знаем, че всеки вектор от $B$ може да се представи като линейна
комбинация на векторите от $A$ и освен това $k > n$, то векторите от системата
$B$ са \acrshort{lin-depen}.
	\sidenote[-1.6cm]{\underline{т.е} ако повече на брой вектори се изразяват линейно чрез
		по-малко на брой вектори, то повечето на брой вектори са \acrshort{lin-depen}.}
\end{lemma}

\begin{definition} \label{dfn:vector-subspace}
Нека $V$ е \acrshort{vecsp} и $W$ е непразно подмножество, $\emptyset \neq W
\subseteq V$. Ще казваме, че $W$ е линейно подпространство (или накратко, само
подпространство) на $V$, ако всяка линейна
комбинация на вектори от $W$ също принадлеци на $W$.
\par Еквивалентно, $W$ е подпространство на $V$, ако е затворено относно
операциите събиране и умножение със скалар.
\sidenote[-.5cm]{Бележим $W < V$.}
\end{definition}

\begin{theorem}[Основни свойства на подпространствата] \label{thm:vector-subspace}
Следните свойства са изпълнени:
\begin{itemize}
\item Всяко подпространство съдържа нулевия вектор.
\item Сечение на подпространства е подпространство.
\item Обединение на подпространства е подпространство само когато едното е
	помножество на другото.
\item Ако $A$ е система от вектори, то $l(A)$ е подпространство (и то
	минималното по включване такова).
\end{itemize}
\end{theorem}

\newpage
\section{Литература}
\nocite{*}
\printbibliography[heading=none]

\end{document}
